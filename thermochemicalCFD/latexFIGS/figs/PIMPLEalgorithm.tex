% !TEX root = ~/OpenFOAM/antoniopucciarelli-9/run/LABS/thermochemical_CFD/main.tex
\begin{figure}[H]

\centering
\begin{tikzpicture}[font=\small, thick, align=center, node distance=0.28cm,x=\textwidth, y=\textheight]

 % Start block
\node[draw,
    rounded rectangle,
    minimum width=2.5cm,
    minimum height=1cm
    ] (start) {START};
 
% getting guess of the solution 
\node[draw,
    trapezium, 
    trapezium left angle = 65,
    trapezium right angle = 115,
    trapezium stretches,
    below=of start,      
    minimum width=3.5cm,  
    minimum height=1cm,
    ] (setting0) { setting $\boldsymbol{u}_0, p_0 $ \\ from \verb|0/| folder};  

% setting up initial values 
\node[draw, 
    below=of setting0,
    ] (settingPar) {setting parameters \\ $\boldsymbol{u}_n = \boldsymbol{u}^* = \boldsymbol{u}_{iter}$ \\ $p_n = p^*$ \\ $t_{n+1} = t_{n} + \Delta t$};

% assembling u predictor step
\node[draw,
    below=0.65cm of settingPar,
    minimum width=3.5cm,
    minimum height=1cm,
    ] (block1) {from equation (\ref{eqn:predictor}), solving \\ $ \boldsymbol{A} \ \boldsymbol{u}^* = \boldsymbol{H}_{(\boldsymbol{u}_{iter})} \ \boldsymbol{u}^* - \boldsymbol{\nabla} p^* + \boldsymbol{f}$ };

% iterative method for u^*
\node[draw, 
    below=of block1, 
    minimum width=3.5cm,
    minimum height=1cm,
    ] (block2) { computing $\boldsymbol{u^*}$ from (\ref{sec:ITER}) \\ $\boldsymbol{M} \ \boldsymbol{u}^*_{j+1} = \boldsymbol{N} \ \boldsymbol{u}^*_j + \boldsymbol{b}_{(p^*, \ \boldsymbol{f})}$ };

% deferred correction step u^*
\node[draw,
    below=of block2,
    ] (deferred1) {deferred correction (\ref{sec:CORR})};

% under relaxation step
\node[draw, 
     below=of deferred1
     ] (Urel1) { $\boldsymbol{u}^*$ under-relaxation (\ref{eqn:PATANKAR})};

% assembling p corrector step
\node[draw, 
    below=0.75cm of Urel1, 
    minimum width=3.5cm,
    minimum height=1cm,
    ] (block3) { from equation (\ref{eqn:corrector}), solving \\ $\boldsymbol{\nabla} \cdot \big( \boldsymbol{A}^{-1} \boldsymbol{H}_{(\boldsymbol{u}_{iter})} \ \boldsymbol{u}^* \big) = \boldsymbol{A}^{-1} \ \boldsymbol{\Delta} p_{n + 1} $ };

% iterative method for p 
\node[draw, 
    below=of block3, 
    minimum width=3.5cm,
    minimum height=1cm,
    ] (block4) {computing $p$ from (\ref{sec:ITER}) \\ $\boldsymbol{M} \ p_{{n + 1}_{j+1}} = \boldsymbol{N} \ p_{{n + 1}_{j}} + \boldsymbol{b}_{(\boldsymbol{u}^*, \ \boldsymbol{f})}$ };

% deferred correction 
\node[draw, 
    below=of block4,
    ] (deferred2) {deferred correction (\ref{sec:CORR})};

% relaxing pressure correction 
\node[draw, 
    below=of deferred2
    ] (nonOrtho) {$p$ non orthogonal corrector (\ref{eqn:nonOrtho})}; 

% correcting velocity u
\node[draw, 
    below=0.5cm of nonOrtho,
    ] (block5) {computing $\boldsymbol{u}_{n + 1}$ from (\ref{eqn:u_comp}) \\ $\boldsymbol{u}_{n + 1} = \boldsymbol{A}^{-1} \boldsymbol{H}_{(\boldsymbol{u}^*)} \ \boldsymbol{u}^* - \boldsymbol{A}^{-1} \boldsymbol{\nabla} p_{n + 1} $};

% relaxation of p and u
\node[draw, 
    below=of block5,
    ] (block6) {$\boldsymbol{u}_{n + 1}$ and $p_{n+1}$ under-relaxation (\ref{eqn:PATANKAR})};

% updating u for the corrector or for the next time iteration 
\node[draw, 
    below=of block6
    ] (block7) {variables updating for corrector \\ $\boldsymbol{u}_{n+1} = \boldsymbol{u}_{iter} = \boldsymbol{u}^*$ \\ $p_{n + 1} = p^*$ };

% 1st bin block
\node[draw,
      fit=(block3) (nonOrtho),
      label=left:\rotatebox{90}{corrector step},
      minimum width = 7cm,
      ] (blockC) {};

% 2nd bin block
\node[draw,
      fit=(block1) (Urel1),
      label=left:\rotatebox{90}{predictor step},
      minimum width = 7cm,
      ] (blockP) {};

% 3rd bin block
\node[draw, 
    fit=(blockC) (block7),
    minimum width=9cm,
    label=left:\rotatebox{90}{internal correctori for $\boldsymbol{u}_{n + 1}$ and $p_{n + 1}$} ] (blockI) {}; 

% 4th bin block     
\node[draw, 
    fit= (blockP) (blockI), 
    label=left:\rotatebox{90}{external corrector for $\boldsymbol{u}_{n + 1}$ and $p_{n + 1}$}, 
    minimum width=11cm,
    minimum height=8cm,
    ] (blockU) {};

% coordinates for arrot
\coordinate[right=2cm of blockU]     (c1)                       {};
\coordinate[]                        (c2) at (c1 |- settingPar) {};
\coordinate[right=0.5cm of nonOrtho] (c3)                       {};
\coordinate[]                        (c4) at (c3 |- block4)     {};
\coordinate[]                        (c5) at (c3 |- Urel1)      {};
\coordinate[]                        (c6) at (c5 |- block2)     {};
\coordinate[right=0.5cm of blockI]   (c7)                       {};
\coordinate[]                        (c8) at (c7 |- blockP)     {};
\coordinate[right=1.2cm of block7]   (c9)                       {};
\coordinate[]                        (c10) at (c9 |- blockC)    {};

% CFL check
\node[draw
    ] (CFLcheck) at (c2) {\verb|PIMPLE| CFL test wrt \verb|controlDict| \\ computing $\Delta t$};

% diamond for choice
\node[diamond, 
    draw, 
    minimum width=3cm, 
    minimum height=1cm, 
    text centered
    ] (error) at (c1 |- deferred1) {if $t_{n + 1} \geq t_{final}$};

% end of the algorithm 
\node[draw, 
    rounded rectangle,
    minimum width = 3.5cm,
    below=5cm of error,
    minimum height=1cm,
    ] (end) at (error.east) {END};

% arrow drawing
\draw[-latex] (start)  to (setting0);
\draw[-latex] (setting0)  to (settingPar);
\draw[-latex] (settingPar) to (blockU);
\draw[-latex] (block1) to (block2);
\draw[-latex] (blockP) to (blockI);
\draw[-latex] (block3) to (block4);
\draw[-latex] (blockU.east) to (c1) to (error);
\draw[-latex] (error.north) to node[anchor=east]  {no}  (CFLcheck);
\draw[-latex] (CFLcheck)    to (settingPar.east);
\draw[-latex] (error.east)  to node[anchor=west] {yes} (end.north);
\draw[-latex] (nonOrtho)  to (c3) to (c4) to (block4);
\draw[-latex] (Urel1)     to (c5) to (c6) to (block2);
\draw[-latex] (block2)    to (deferred1);
\draw[-latex] (deferred1) to (Urel1);
\draw[-latex] (block4)    to (deferred2);
\draw[-latex] (deferred2) to (nonOrtho);
\draw[-latex] (block5)    to (block6);
\draw[-latex] (block6)    to (block7);
\draw[-latex] (blockI)    to (c7) to (c8) to (blockP);
\draw[-latex] (blockC)    to (block5);
\draw[-latex] (block7)    to (c9) to (c10) to (blockC);

\end{tikzpicture}
    \caption{$\mathtt{PIMPLE}$ algorithm.}
    \label{alg:PIMPLE}

\end{figure}
