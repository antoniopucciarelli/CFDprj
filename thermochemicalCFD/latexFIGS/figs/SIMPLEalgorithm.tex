% !TEX root = ~/OpenFOAM/antoniopucciarelli-9/run/LABS/thermochemical_CFD/main.tex
\begin{figure}[H]

\centering
\begin{tikzpicture}[font=\small, thick, align=center, node distance=0.4cm, remember picture, x=\textwidth, y=\textheight]

 % Start block
\node[draw,
    rounded rectangle,
    minimum width=2.5cm,
    minimum height=1cm
    ] (start) {START};
 
% getting guess of the solution 
\node[draw,
    trapezium, 
    trapezium left angle = 65,
    trapezium right angle = 115,
    trapezium stretches,
    below=of start,      
    minimum width=3.5cm,  
    minimum height=1cm,
    ] (guess) { guessing $\boldsymbol{u}, p $ \\ from \verb|0/| folder};  

% setting up initial values 
\node[draw, 
    below=of guess,
    ] (settingPar) {setting parameters \\ $\boldsymbol{u} = \boldsymbol{u}^*$ \\ $p = p^* $};

% assembling u predictor step
\node[draw,
    below=of settingPar,
    minimum width=3.5cm,
    minimum height=1cm,
    ] (block1) {from equation (\ref{eqn:predictor}), solving \\ $ \boldsymbol{A} \ \boldsymbol{u}^* = \boldsymbol{H}_{(\boldsymbol{u})} \ \boldsymbol{u}^* - \boldsymbol{\nabla} p^* + \boldsymbol{f}$ };

% iterative method for u^*
\node[draw, 
    below=of block1, 
    minimum width=3.5cm,
    minimum height=1cm,
    ] (block2) { computing $\boldsymbol{u^*}$ from (\ref{eqn:ITER}) \\ $\boldsymbol{M} \ \boldsymbol{u}^*_{j+1} = \boldsymbol{N} \ \boldsymbol{u}^*_j + \boldsymbol{b}_{(p^*, \ \boldsymbol{f})}$ };

% deferred correction step u^*
\node[draw,
    below=of block2,
    ] (deferred1) {deferred correction (\ref{sec:CORR})};

% under relaxation step
\node[draw, 
     below=of deferred1
     ] (Urel1) { $\boldsymbol{u}^*$ under-relaxation (\ref{eqn:PATANKAR})};

% assembling p corrector step
\node[draw, 
    below=0.8cm of Urel1, 
    minimum width=3.5cm,
    minimum height=1cm,
    ] (block3) { from equation (\ref{eqn:corrector}), solving \\ $\boldsymbol{\nabla} \cdot \big( \boldsymbol{A}^{-1} \boldsymbol{H}_{(\boldsymbol{u})} \ \boldsymbol{u}^* \big) = \boldsymbol{A}^{-1} \ \boldsymbol{\Delta} p $ };

% iterative method for p 
\node[draw, 
    below=of block3, 
    minimum width=3.5cm,
    minimum height=1cm,
    ] (block4) {computing $p$ from (\ref{eqn:ITER}) \\ $\boldsymbol{M} \ p_{j+1} = \boldsymbol{N} \ p_j + \boldsymbol{b}_{(\boldsymbol{u}^*, \ \boldsymbol{f})}$ };

% deferred correction 
\node[draw, 
    below=of block4,
    ] (deferred2) {deferred correction (\ref{sec:CORR})};

% relaxing pressure correction 
\node[draw, 
    below=of deferred2
    ] (nonOrtho) {$p$ non orthogonal corrector (\ref{eqn:nonOrtho})}; 

% p under relaxation 
\node[draw, 
    below=0.65cm of nonOrtho
    ] (Urel2) { $p$ under relaxation (\ref{eqn:PATANKAR})}; 

% correcting velocity u
\node[draw, 
    below=of Urel2
    ] (block5) {computing $\boldsymbol{u}$ from (\ref{eqn:u_comp}) \\ $\boldsymbol{u} = \boldsymbol{A}^{-1} \boldsymbol{H}_{(\boldsymbol{u}^*)} \ \boldsymbol{u}^* - \boldsymbol{A}^{-1} \boldsymbol{\nabla} p $ };

% 1st bin block
\node[draw,
      fit=(block3) (nonOrtho),
      label=left:\rotatebox{90}{corrector step},
      minimum width = 7cm,
      ] (blockC) {};

% 2nd bin block
\node[draw,
      fit=(block1) (Urel1),
      label=left:\rotatebox{90}{predictor step},
      minimum width = 7cm,
      ] (blockP) {};

% 3rd bin block
\node[draw, 
    fit=(Urel2) (block5), 
    label=left:\rotatebox{90}{$\boldsymbol{u}$ update}, 
    minimum width=7cm,
    ] (blockU) {};

% 4th bin block 
\node[draw, 
    fit= (blockC) (blockU),
    label=left:\rotatebox{90}{internal corrector for $\boldsymbol{u}$ and $p$},
    minimum width = 9cm,
    ] (blockI) {}; 

% coordinates for arrotw
\coordinate[right=2.5cm of blockI]   (c1)                       {};
\coordinate[]                        (c2) at (c1 |- settingPar) {};
\coordinate[right=0.5cm of nonOrtho] (c3)                       {};
\coordinate[]                        (c4) at (c3 |- block4)     {};
\coordinate[]                        (c5) at (c3 |- Urel1)      {};
\coordinate[]                        (c6) at (c5 |- block2)     {};
\coordinate[right=0.5cm of blockU]   (c7)                       {};
\coordinate[]                        (c8) at (c7 |- blockC)     {};

% diamond for choice
\node[diamond, 
    draw, 
    minimum width=3cm, 
    minimum height=1cm, 
    text centered
    ] (error) at (c1 |- block3) {error check \\ residual (\ref{sec:ITER}) \\ iteration number};

% end of the algorithm 
\node[draw, 
    rounded rectangle,
    below=5cm of error,
    minimum width = 3.5cm,
    minimum height=1cm,
    ] (end) at (error.east) {END};

% arrow drawing
\draw[-latex] (start)  to (guess);
\draw[-latex] (guess)  to (settingPar);
\draw[-latex] (settingPar) to (blockP);
\draw[-latex] (block1) to (block2);
\draw[-latex] (blockP) to (blockI);
\draw[-latex] (block3) to (block4);
\draw[-latex] (blockC) to (blockU);
\draw[-latex] (Urel2)  to (block5);
\draw[-latex] (blockI.east) to (c1) to (error);
\draw[-latex] (error.north) -- node[anchor=east]  {no} (c2) to (settingPar.east);
\draw[-latex] (error.east)  -- node[anchor=west] {yes} (end.north);
\draw[-latex] (nonOrtho)  to (c3) to (c4) to (block4);
\draw[-latex] (Urel1)     to (c5) to (c6) to (block2);
\draw[-latex] (block2)    to (deferred1);
\draw[-latex] (deferred1) to (Urel1);
\draw[-latex] (block4)    to (deferred2);
\draw[-latex] (deferred2) to (nonOrtho);
\draw[-latex] (blockU)    to (c7) to (c8) to (blockC);

\end{tikzpicture}
    \caption{$\mathtt{SIMPLE}$ algorithm.}
    \label{alg:SIMPLE}

\end{figure}
